\documentclass[12pt]{article}
\textwidth 15.4 cm
\textheight 21.5 cm
\topmargin 0.5cm
\evensidemargin 2 cm
\oddsidemargin 3 mm

\newcommand{\qed}{\rule[-0.2ex]{0.3em}{1.4ex}}
\newcommand{\out}[1]{\mbox{\bf {\sf Out}}({#1})}
\newenvironment{Proof}%
{\par\vspace{0.5ex}\noindent{\bf Proof:}\hspace{0.5em}}%
{\nopagebreak
\strut\nopagebreak
\hspace{\fill}\qed\par\medskip\noindent}
\newenvironment{proof}% lowercase version...
{\par\vspace{0.5ex}\noindent{\bf Proof:}\hspace{0.5em}}%
{\nopagebreak
\strut\nopagebreak
\hspace{\fill}\qed\par\medskip\noindent}
\newenvironment{proofattempt}%
{\par\vspace{0.5ex}\noindent{\bf Proof Attempt:}\hspace{0.5em}}%
{\nopagebreak
\strut\nopagebreak
\hspace{\fill}\qed\par\medskip\noindent}
\newtheorem{theorem}{Theorem}
\newtheorem{claim}{Claim}
\newtheorem{prop}{Proposition}
\newtheorem{property}{Property}
\newtheorem{lemma}{Lemma}
\newtheorem{fact}{Fact}
\newtheorem{definition}{Definition}
\newtheorem{corollary}{Corollary}
\newtheorem{remark}{Remark}
\newtheorem{observation}{Observation}
\newtheorem{conjecture}{Conjecture}

\newcommand{\support}{\mbox{\sf supp}}
\newcommand{\OMS}{\mbox{$\bf \Omega_{S}$}}

%\definecolor{ao(english)}{rgb}{0.0, 0.5, 0.0}

\usepackage{amsmath}
\usepackage{amsfonts}
\usepackage{color}
\usepackage{circuitikz}
\usepackage{listings}
\usepackage{circuitikz}
%\usepackage{tikz}


\usepackage{textcomp}


\usepackage{listings}
\lstset{escapeinside={<@}{@>}}
\usepackage{color}

\definecolor{darkgreen}{rgb}{0.31,0.45,0.25}

\usepackage{textcomp}


\usepackage[T1]{fontenc}
\usepackage{inconsolata}

\usepackage{color}

\definecolor{pblue}{rgb}{0.13,0.13,1}
\definecolor{pgreen}{rgb}{0,0.5,0}
\definecolor{pred}{rgb}{0.9,0,0}
\definecolor{pgrey}{rgb}{0.46,0.45,0.48}


\lstset{language=Java,
  showspaces=false,
  showtabs=false,
  breaklines=true,
  showstringspaces=false,
  breakatwhitespace=true,
  commentstyle=\color{pgreen},
  keywordstyle=\color{pblue},
  stringstyle=\color{pred},
  basicstyle=\ttfamily,
  moredelim=[il][\textcolor{pgrey}]{$$},
  moredelim=[is][\textcolor{pgrey}]{\%\%}{\%\%}
}



\lstset{
	language=Java
}




\usepackage{textcomp}
\usepackage{booktabs}


\DeclareMathOperator*{\argmax}{arg\,max}
\DeclareMathOperator*{\argmin}{arg\,min}
\DeclareMathOperator*{\cov}{\sf cov}
 
\newcommand{\IR}{{\rm\hbox{I\kern-.15em R}}}
\newcommand{\reals}{{\rm\hbox{I\kern-.15em R}}}
\newcommand{\IN}{{\rm\hbox{I\kern-.15em N}}}
\newcommand{\IZ}{{\sf\hbox{Z\kern-.40em Z}}}
\newcommand{\id}{\mbox{\bf \em id}}
\newcommand{\lo}{\mbox{\bf \em loc}}
\newcommand{\E}{\mbox{\sf {I\kern-.15em E}}}
\newcommand{\F}{\mbox{\bf F}}
\newcommand{\Var}{\mbox{\bf Var}}
\newcommand{\Prob}{\mbox{\sf \hbox{I\kern-.15em P}}}
\newcommand{\g}{\mbox{\bf \em g}}
\newcommand{\SeqRank}{\mbox{\bf \em SequentialRank}}

\newcommand{\vat}{\mbox{\scshape \tiny VAT}}
\newcommand{\emm}{\mbox{\scshape \tiny EM}}
\newcommand{\ram}{\mbox{\scshape \tiny RAM}}
\newcommand{\mNC}{\ensuremath{\mbox{m}{\cal NC}^{1}}}

\newcommand{\ta}{\mbox{\color{blue} T}}
\newcommand{\h}{\mbox{\color{red} H}}

%
%
%
%\lstset{language=Scheme} 
%
%
%


\begin{document}
\title{{\bf The Java Singleton Pattern\\DRAFT}}
\author{
Phillip G. Bradford\thanks{phillip.bradford@uconn.edu, phillip.g.bradford@gmail.com,
{\sc University of Connecticut, Department of Computer Science and Engineering, Storrs, CT USA}}
}

\date{\small\today}

\maketitle

%
%
%
\begin{abstract}
The Java singleton pattern is to ensure a JVM contains only one instance of an object.
The version here is a common implementation that depends on a static field and a static method.
 % .
\end{abstract}

%
%
%
%
\section{Java Singleton}
\label{Java Singleton}
%
%
%

The next code is in \lstinline|https://github.com/wonder-phil/a2021_04_09_singleton|


\begin{lstlisting}[label=JavaSingleton,frame=lines,caption=Basic Singleton]
public class Singleton {
	
	private Singleton() { }
	
	private static Singleton CLASS_INSTANCE = new Singleton();

	public static Singleton getInstance() {
		return CLASS_INSTANCE;
	}
}
\end{lstlisting}

Listing~\ref{JavaSingleton} shows a classical implementation of a Java singleton depending on Java's static functionality.
Particularly, the \lstinline|private| access modifier for the constructor prevents this class from being instantiated by any other
class.
The \lstinline|private static| modifier for the CLASS\_INSTANCE field ensures the one CLASS\_INSTANCE here is associated with the class
rather than any object instance.

\begin{lstlisting}[label=JavaNONSingleton,frame=lines,caption=Basic Non-Singleton]
public class NonSingleton {
	
	public NonSingleton() { }
	private int value = 0;
}
\end{lstlisting}

\begin{lstlisting}[label=JavaTESTSingleton,frame=lines,caption=Basic Singleton and Non-Singleton Test]
public class TestSingleton {
	
	public static void main(String[] args) {
		
		Singleton so_1 = Singleton.getInstance();
		Singleton so_2 = Singleton.getInstance();

		System.out.print("SingletonObject: ");
		if (so_1 == so_2) {
			System.out.println("Same object instances");
		} else {
			System.out.println("Different object instances");
		}
		
		NonSingleton ns_1 = new NonSingleton();
		NonSingleton ns_2 = new NonSingleton();
		
		System.out.print("Nonsingleton: ");
		if (ns_1 == ns_2) {
			System.out.println("Same object instances");
		} else {
			System.out.println("Different object instances");
		}
	}
}
\end{lstlisting}

%
%
\pagebreak
%
%

%
%
\section{Exercises}

%
%
%


\begin{enumerate}

\item
What is the difference between a class-variable and an instance-variable?

\item
Try inheriting from the instance of the \lstinline|Singleton| class.
Is this possible?

\item 
How might the instance variable \lstinline|value| be useful?



\end{enumerate}

%

%
%
%
\pagebreak
%
%

%
\begin {thebibliography}{99}
%
 
 \bibitem{JavaOracle} Java The Complete Reference, Ninth Edition (Java 8),
	Herbert Schildt, Oracle Press/McGraw-Hill, 2014.


\end {thebibliography}
\end{document}